\begin{frame}
  \frametitle{Características mais marcantes}

  \begin{itemize}
    \item Como dito anteriormente, a linguagem só faz utilização de funções e funções dentro de funções. Por isso
    Haskell é descrito como puramente funcional. 
    \item Haskell possui uma sintaxe simples, elegante e concisa. Como resultado, programas em Haskell possuem 
    poucas linhas. 
    \item Além disso, a linguagem usa avaliação preguiçosa(Lazy evaluation), que é uma técnica para atrasar a computação 
    até um ponto em que o resultado da computação é considerado necessário.
    \item Tipagem estática: Verificação dos tipos usados em dados e variáveis para 
    garantir que sempre está sendo usado um tipo que é esperado em todas as situações. 
    \item Função de ordem superior: Função que tem como argumento uma outra função, ou que produz 
    uma função como resultado.
    \item Antes da execução de um programa, os compiladores e interpretadores realizam uma checagem forte de tipos
    de dados, verificação monomórfica e verificação polimórfica.
  \end{itemize}

\end{frame}
