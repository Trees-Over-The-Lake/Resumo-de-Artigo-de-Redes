\documentclass[aspectratio=169]{beamer}	 	

\usetheme{Goettingen}
\usecolortheme{default}
\usefonttheme{professionalfonts}			% para fontes matemáticas
% Enconte mais temas e cores em http://www.hartwork.org/beamer-theme-matrix/ 
% Veja também http://deic.uab.es/~iblanes/beamer_gallery/index.html

% Customizações de Cores: fg significa cor do texto e bg é cor do fundo

\definecolor{mypurple}{RGB}{75, 0, 130} % changed this
\definecolor{mypurple2}{RGB}{44, 0, 73} % changed this
\definecolor{mygray}{RGB}{250, 242, 250} % changed this
\definecolor{myblue}{RGB}{0, 0, 139} % changed this
\definecolor{baguete}{RGB}{255, 252, 255} % changed this

\setbeamercolor{normal text}{fg=black}
\setbeamercolor{alerted text}{fg=red}
\setbeamercolor{author}{fg=blue}
\setbeamercolor{institute}{fg=blue}
\setbeamercolor{date}{fg=myblue}
\setbeamercolor{frametitle}{fg=mypurple}
\setbeamercolor{framesubtitle}{fg=black}
\setbeamercolor{block title}{bg=mypurple2, fg=white}		%Cor do título
\setbeamercolor{block body}{bg=mygray, fg=darkgray}	%Cor do texto (bg= fundo; fg=texto)
\setbeamercolor{structure}{fg=mypurple}

\setbeamercolor{background canvas}{bg=baguete}

% informações do PDF
\makeatletter
\hypersetup{
     	%pagebackref=true,
		pdftitle={\@title}, 
		pdfauthor={\@author},
    	pdfsubject={Trabalho de Linguagem de programação em LaTeX},
	    pdfcreator={GLR, LSO, PS, THN},
		pdfkeywords={abnt}{latex}{abntex}{abntex2}{Linguagem de programação}, 
		colorlinks=true,       		% false: boxed links; true: colored links
    	linkcolor=black,          	% color of internal links
    	citecolor=blue,        		% color of links to bibliography
    	filecolor=magenta,      		% color of file links
		urlcolor=blue,
		bookmarksdepth=4
}
\makeatother

% ---
% PACOTES
% ---
\usepackage[alf]{abntex2cite}	% Citações padrão ABNT
\usepackage[brazil]{babel}		% Idioma do documento
\usepackage{color}			      % Controle das cores
\usepackage[T1]{fontenc}		  % Selecao de codigos de fonte.
\usepackage{graphicx}			    % Inclusão de gráficos
\usepackage[utf8]{inputenc}		% Codificacao do documento (conversão automática dos acentos)
\usepackage{txfonts}			    % Fontes virtuais
\usepackage{import}
% ---

% --- Informações do documento ---
\title{Apresentando Haskell}
\author{Gustavo Lopes \and Lucas Santiago \and Pedro Souza \and Thiago Henriques }
\institute{Pontifícia Universidade Católica de Minas Gerais}
\date{26 de Março de 2021}
% ---

% ----------------- INÍCIO DO DOCUMENTO --------------------------------------
\begin{document}

    % ----------------- INICIO --------------------------------
    \import{./seccoes}{primeiro_slide.tex}

    \import{./seccoes}{tabela_de_conteudos.tex}

    % ----------------- INTRODUCAO --------------------------------

    \section{Introdução}

    \import{./seccoes/introducao}{introducao.tex}

    \import{./seccoes/historico_da_linguagem}{versoes_lancadas.tex}

    \import{./seccoes/historico_da_linguagem}{versoes_lancadas2.tex}

    \import{./seccoes/historico_da_linguagem}{haskell_atualmente.tex}

    % ----------------- PARADIGMA --------------------------------
    \section{Paradigma}

    \import{./seccoes/paradigma}{definicao.tex}

    \import{./seccoes/paradigma}{imperativo_vs_declarativo.tex}
    
    \import{./seccoes/paradigma/funcional}{definicao.tex}

    \import{./seccoes/paradigma/funcional}{caracteristicas.tex}

    \import{./seccoes/paradigma/funcional}{analise_critica.tex}

    % ----------------- CARACTERISTICAS MAIS MARCANTES --------------------------------

    \section{Características mais marcantes}

    \import{./seccoes/caracteristicas_marcantes}{caracteristicas_marcantes.tex}

    % ----------------- RELACAO COM OUTRAS LINGUAGENS --------------------------------

    \section{Relacao com outras linguagens}

    \import{./seccoes/relacao_com_outras_linguagens}{relacao_com_outras_linguagens.tex}

    % ----------------- EXEMPLO DE PROGRAMAS --------------------------------

    \section{Exemplo(s) de programa(s)}

    \import{./seccoes/exemplo_de_programas}{exemplo_de_programas.tex}

    % ----------------- CONCLUSAO --------------------------------
    \section{Considerações finais}

    \import{./seccoes/conclusao}{conclusao.tex}
    
    % ----------------- REFERENCIAS --------------------------------
    \section{Referências}

    \import{./seccoes/referencias}{referencias.tex}

% ----------------- FIM DO DOCUMENTO -----------------------------------------
\end{document}
